\documentclass[fontsize=11pt]{article}
\usepackage{amsmath}
\usepackage[T1]{fontenc}
\usepackage[utf8]{inputenc}
\usepackage[margin=0.75in]{geometry}
\usepackage{hyperref}
\urlstyle{same}

\title{CSC111 Project 2 Proposal: \\ \textbf{“Destiny” - Find Your Next Soulmate Here}}
\author{Hsin Kuang (Joseph) Cheung \\ Tsz Kan Charlotte Wong \\ Yan Lam \\ Chun Yin Liang}
\date{\today}

\begin{document}
\maketitle

\section*{Problem Description and Research Question}

The University of Toronto is a community with a diverse student population, with lots of international students (University of Toronto, 2024).
As we enter early adulthood, the struggle between isolation and intimacy becomes significant.
Many students feel lonely and stressed when making new connections and managing their academic burden, especially those from abroad who experience homesickness.
\textbf{How can students in UofT meet new people and build a supportive circle beyond lectures with less effort?
This is a challenge that is faced by most of our group members since we are new to the city.}
\\
\\
To address this problem, we wanted to introduce a new application called \textbf{Destiny}.
It is designed to make socializing easier and more interesting, enriching students’ university lives.
This is not just another “dating app” but a platform where students can seek both friends and potential romantic partners. By offering a wide pool of potential matches, the application helps users connect effortlessly - whether you are looking for a study partner, a coffee companion, or even a date.
\\
\\
With a user-friendly interface, students can create profiles showcasing their interests, personality traits, and preferred connections.
The algorithm helps suggest matches based on shared hobbies and values, making it easier to meet like-minded people while reducing the insecurity of approaching someone out of fear of rejection. To ensure more compatible connections, the application provides a personalized environment where students can select their preferred qualities.
\\
\\
\textbf{By encouraging friendships and romantic bonds, we aim to create a welcoming platform where students can develop personally, and build meaningful, enduring relationships.}
Four of our group members are international students themselves, and thus, we understand firsthand the importance of a strong supportive network, either as friends or something more, in enhancing mental health and student engagement throughout university life.


\section*{Computational Plan}

Ideally, we should collect data from current UofT students to build a realistic database of actual users.
However, due to time constraints and privacy concerns, we will generate a synthetic dataset of 200 users for this project.
These users will have different attributes and preferences, and we will use the Faker Python library for generating data.
\\
\\
We are implementing a Binary Tree to match users to generate a list of matches based on the users’ preference for attributes.
Before using the app, each user is asked to choose whether they are looking for: friends, potential romantic partners, or both.
Two trees can be generated based on the user’s answer: one for matching friends, and one for matching potential romantic partners.
Then, the user is asked to enter their answer for each attribute.
Each user ranks their priorities for different attributes (e.g., common interests, program of study, personality traits).
\\
\\
According to their priorities, a binary decision tree is built using these ranked attributes as decision nodes.
For example, if a user prioritizes hobbies the most, then hobbies will be at the first level of the tree.
Then, the most important attribute is at the root of the tree, and the least important attribute is at the bottom of the tree.
The root of the tree is “ ”, and each node could be either: \texttt{Yes} (match), or \texttt{No} (mismatch), except the leaves where nodes represent users.
The program will systematically traverse through the tree (taking the \texttt{Yes} path).
After reaching a leaf, the user will be added to \texttt{result}, which is a list of potential matches in order.
If there is more than one leaf with common attributes, start from the leftmost node.
(For example, if user A and user B share the same attributes, user A will be added to \texttt{result} first.)
After the first all-\texttt{Yes} path, the second priority would be given to the path where the last level is \texttt{No}
(which means the least important attribute is mismatched, but other higher-priority attributes are matched).
The final full list \texttt{result}, including all users, is generated and ordered according to ranking based on compatibility.
\\
\\
The user will be first shown with the first matching user in the \texttt{result} \textt{list} (i.e., \texttt{result[0]}).
If the user “refreshes” the page, then the next matching user (\texttt{result[1]}) will be suggested, and so on.
If the user successfully matches with another user, they will be stored in \texttt{romantic\_matches} and/or \texttt{friend\_matches},
which are lists of sets, with the data type of \texttt{list[set[User]]}.
Each set contains two users, which are instances of the \texttt{User} class.
Also, \texttt{user\_list} is a list of all users in the system, with the data type of \texttt{list[User]}.
\\
\\
After users are matched through the system, they can be represented as a graph.
Network graphs will be used to visualize the connections between all users.
There will be two graphs, one targeted at people looking for romantic relationships and one for friendships.
Each node represents one user, and each edge represents a match between two users.
The size of each node is directly proportional to its degree, i.e., a user with more friends will appear as a larger node in the graph.
To distinguish between current and past relationships, the edges for current couples and friends will be in red, while the edges for ex-couples and ex-friends will be in blue.
\\
\\
These two network graphs are created from \texttt{user\_list} (for generating all the nodes on the graphs),
\texttt{romantic\_matches}, and \texttt{friend\_matches} (for generating the edges, and showing their connections).
We will generate these interactive graphs using plotly, an interactive graphing Python library (Plotly, 2025).
Then, as the developers of the application, we can zoom in and out of the graph, and view the detailed information of different users in our system.

\section*{References}

Plotly. (2025). \textit{Plotly}.
\url{https://plotly.com/python/}
\\
University of Toronto. (2024). \textit{Quick facts}.
\url{https://www.utoronto.ca/about-u-of-t/quick-facts}


% NOTE: LaTeX does have a built-in way of generating references automatically,
% but it's a bit tricky to use so we STRONGLY recommend writing your references
% manually, using a standard academic format like APA or MLA.
% (E.g., https://owl.purdue.edu/owl/research_and_citation/apa_style/apa_formatting_and_style_guide/general_format.html)

\end{document}
