\documentclass[fontsize=11pt]{article}
\usepackage{amsmath}
\usepackage[utf8]{inputenc}
\usepackage[margin=0.75in]{geometry}

\title{CSC111 Project 2 Proposal: TODO FILL IN YOUR PROJECT TITLE HERE}
\author{Hsin Kuang (Joseph) Cheung \\ Tsz Kan Charlotte Wong \\ Yan Lam \\ Chun Yin Liang}
\date{\today}

\begin{document}
\maketitle

\section*{Problem Description and Research Question}

hihi
test

\section*{Computational Plan}

We are implementing a Binary Tree to match users to generate a list of matches based on the users’ preference of attributes. Before using the app, each user is asked to choose whether they are looking for: friends, potential romantic partners, or both.
Two trees can be generated based on the user’s answer: one for matching friends, and one for matching potential romantic partners.
Then, the user is asked to enter their answer for each attribute.
Each user ranks their priorities for different attributes (e.g., common interests, program of study, personality traits).
\\
\\
According to their priorities, a binary decision tree is built using these ranked attributes as decision nodes.
For example, if a user prioritizes hobbies the most, then hobbies will be at the first level of the tree.
The root of the tree is ‘’, and each node could be either: \texttt{Yes} (match), or \texttt{No} (mismatch).
The tree systematically goes to potential matches (\textbf{Yes} path) based on the highest-priority attribute first through moving down the tree.
Each node will be added into a list \texttt{<result>} accordingly.
\\
\\
If there are more than one node with common attributes, go from the leftmost node. (For example, if A and B share the same attributes, A goes first.)
After the first all-\textbf{Yes} path, the second priority would be given to the path where the last level is \textbf{No} (which means the least important attribute is mismatched, but other higher-priority attributes are matched).
The final full list, including all users, is generated and ordered according to ranking based on compatibility.
The user will be first shown with \texttt{index[0]} of \texttt{<result>}.
If the user “refreshes” the page, \texttt{index[1]} will be suggested, and so on.
\\
\\
After users are matched through the system, they can be represented as a graph. Network graphs will be used to visualize the connections between all users. There will be two graphs, one targeted at people looking for romantic relationships and one for friendships. Each node represents one user, and each edge represents a match between two users. The size of each node is directly proportional to its degree, i.e., a user with more friends will appear as a larger node in the graph. To distinguish between current and past relationships, the edges for current couples and friends will be in red, while the edges for ex-couples and ex-friends will be in blue.
\\
\\
These two network graphs are created from the list of users and the matches between them if any. We will use plotly to generate these interactive graphs, where we can zoom in and out of the graph, and view the information of a selected user.

\section*{References}

(Joseph et al., 2025)

% NOTE: LaTeX does have a built-in way of generating references automatically,
% but it's a bit tricky to use so we STRONGLY recommend writing your references
% manually, using a standard academic format like APA or MLA.
% (E.g., https://owl.purdue.edu/owl/research_and_citation/apa_style/apa_formatting_and_style_guide/general_format.html)

\end{document}
